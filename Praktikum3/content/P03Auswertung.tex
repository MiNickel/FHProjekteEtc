\newcommand{\env}[1]{\texttt{#1}}
\newcommand{\command}[1]{\texttt{#1}}
\newcommand{\package}[1]{\texttt{\itshape#1}}
\newcommand{\engl}[1]{(engl: \textit{#1})\xspace}
\newpage

\section{Aufgabe 1}

\subsection{Frage- bzw. Aufgabenstellung}
Eine der bekanntesten Applikationen im Internet ist eMail. Erklären Sie in Stichworten, welche Protolle und Abläufe das Senden und Empfangen von eMails ermöglichen. Wie funktioniert das SMTP Protokoll? Wie funktionieren die Protokolle POP3 und IMAP?


\subsection{Lösung}
\subsubsection{SMTP} 
Das SMTP Protokoll funktioniert, indem der MTA (message transfer agent) vom Sender eine TCP-Verbindung mit dem MTA des Empfängers aufbaut. Das eigentliche Senden der Mail funktioniert dann mit der Übertragung von Anfragen in Form von 5 essentiellen Befehlen: \\
\\
- HELLO <Hostname> : Client übermittelt dem Server seine  Identität in Form des FQDN (fully qualified domain name) \\
- MAIL FROM <Adresse> : Festlegung des Senders \\
- RCPT TO <Adresse> : Festlegung des Empfängers \\
- DATA : Einleitung der Übertragung des Nachrichtinhaltes \\
- QUIT : Beenden der Verbindung \\
\\
Der MTA des Empfängers kann auf diese Anfragen entsprechend reagieren, entweder nimmt er die Anfragen an und akzeptiert die Übertragung oder er lehnt diese ab. \cite{[1]}

\subsubsection{POP3} 
Das POP Protokoll wird verwendet um E-Mails mit Hilfe eines Client vom Server abzurufen. Dieses Protokoll benutzt mehrere Befehle um die Mails abzurufen: \\
\\
- USER : Identifiziert die Mailbox(User)\\
- PASS : Authentifizierung des Users mit Hilfe eines Passwortes\\
- QUIT :  Beendung der Verbindung zum Server\\
- STAT : Anforderung der Anzahl der vorhandenen E-Mails\\
- LIST : Anforderung der Nummer und der Größe aller E-Mails\\
- RETR : E-Mail vom Server anfordern\\
- DELE : Angeforderte E-Mail wird gelöscht\\
- NOOP : Aufrechterhaltung der Verbindung (verhindern von Timeouts)\\
- RSET : Zurücksetzen der aktuellen Verbindung\\
\\
Auch hier antwortet der Server entsprechend auf die Anfragen, indem er sie akzeptiert oder sie ablehnt. \cite{[2]}

\subsubsection{IMAP} 
Das IMAP Protokoll ist eine Art Erweiterung des POP3 Protokolls. Im Gegensatz zum POP3 Protokoll, wartet IMAP nicht auf eine Antwort des Server, sondern sendet direkt mehrere Befehle hintereinander mit einer Kennung. Diese Kennung wird bei der Antwort des Servers mit zurückgeliefert. In der Antwort des Servers steht dann, ob Befehl ein Erfolg oder Misserfolg war und ob der Server noch weitere Informationen erwartet oder ob der Server dem Client die Informationen schickt. \cite{[3]} 

\subsection{Ergebnis}
Wir kennen die Protokolle, die für den Versand von E-Mails verwendet werden und wissen wie diese funktionieren.

\section{Aufgabe 2}

\subsection{Frage- bzw. Aufgabenstellung}
Schreiben Sie ein eigenes Java-Programm, das Mails versendet. Verwenden Sie dazu die Bibliothek javamail (javamail.java.net). 

\subsection{Vorbereitung}
Zur Vorbereitung muss die jeweilige Bibliothek installiert werden und es muss sich in die Bibliothek eingearbeitet/eingelesen werden.

\subsection{Aufgabenteil a)}
\subsubsection{Frage- bzw. Aufgabenstellung}

Verwenden Sie als SMTP Ausgangsserver einen Ihrer eigenen Mailaccounts, z.B. den der FH Bielefeld. Die Konfiguration kennen Sie aus den Hilfeseiten der DVZ. Das Programm soll in der Lage sein, eine Mail mit einem von ihnen vorgegebenen Inhalt an eine beliebige Mailadresse zu senden. Der Inhalt der Mail soll aus einer Datei inhalt.txt aus dem Dateisystem Ihres Rechners eingelesen werden.

\subsubsection{Lösung}
Damit das Programm E-Mails verschicken kann, müssen vorher Konfigurationen (Properties) festgelegt werden. Diese bestimmen das verwendete Protokoll (siehe Zeile 2), den Host-Server (siehe Zeile 3), den verwendeten Port (siehe Zeile 4), ob eine Authentifizierung stattfinden soll (siehe Zeile 5) und dass TLS benutzt wird (siehe Zeile 6). \cite{[5,4]}
\begin{lstlisting}
		Properties props = new Properties();
		props.setProperty("mail.transport.protocol", "smtp");
		props.setProperty("mail.smtp.host", "smtp.fh-bielefeld.de");
		props.setProperty("mail.smtp.port", "587");
		props.setProperty("mail.smtp.auth", "true");
		props.setProperty("mail.smtp.starttls.enable", "true");
\end{lstlisting}


Zur Authentifizierung wird folgendes benötigt. Zum einen muss man die Session mit den zuvor bestimmten Konfigurationen erzeugen. Des Weiteren fügt man der Session einen Authenticator hinzu, der zur Abfrage des Users und des Passwortes ist (siehe Zeile 2-3). \cite{[6]}
\begin{lstlisting}
		Session session = Session.getInstance(props, new javax.mail.Authenticator() {
			protected PasswordAuthentication getPasswordAuthentication() {
				return new PasswordAuthentication("user", "password");
			}
		});
\end{lstlisting}


Zum Erstellen der E-Mail wird ein Objekt von Typ Message erstellt(Zeile 1). Diesem Objekt gibt man dann die EMail des Sender(Zeile 2) und des Empfängers(Zeile 3), einen Betreff(Zeile 4) und einen Inhalt (Zeile 5). Zum Senden der E-Mail benutzt man den Befehl Transport.send (siehe Zeile 7).
\begin{lstlisting}
			Message message = new MimeMessage(session);
			message.setFrom(new InternetAddress(Sender Email));
			message.setRecipients(Message.RecipientType.TO, InternetAddress.parse(Empfaenger Email));
			message.setSubject("Betreff");
			message.setText("Inhalt");
			
			Transport.send(message);
\end{lstlisting}


Da der Inhalt aus eine Text-Datei genommen werden soll, muss man diese vorher einlesen(Zeile 1-2). Der String der durch das Lesen der Datei entsteht wird dann der jeweiligen Methode zum setzen des E-Mail Inhaltes übergeben(Zeile 4).
\begin{lstlisting}
		ReadFile contentReader = new ReadFile();
		String mailText = contentReader.readFile("C:/Users/Jan Augstein/eclipse-workspace/VSPraktikum3Mail/src/inhalt.txt");
		
		message.setText(mailText);
\end{lstlisting}

\subsubsection{Ergebnis}
Man hat ein Programm zum Versenden von E-Mails, dessen Inhalt aus einer Text-Datei gelesen wird. (Umgang mit javamail)

\subsection{Aufgabeteil b)}
\subsubsection{Frage- bzw. Aufgabenstellung}
 Erweitern Sie ihr Programm so, dass automatisch eine Liste von Empfängern aus der Datei empfaenger.txt eingelesen wird.
 
\subsubsection{Lösung}
Man liest die Empfaenger.txt Datei ein und teilt den Inhalt so auf ein String Array auf, sodass hinter einem Index des Arrays immer eine E-Mail steht. 
\begin{lstlisting}
		ReadFile recipientReader = new ReadFile();
		String recipList = recipientReader.readFile("C:/Users/Jan Augstein/eclipse-workspace/VSPraktikum3Mail/src/empfaenger.txt");
		String recipient[] = recipList.split("\n");
\end{lstlisting} 


Diese E-Mails werden dann abgearbeitet und es wird an jede E-Mail-Adresse die E-Mail geschickt.
\begin{lstlisting}
		for(int i=0;i<recipient.length;i++) {
				message.setRecipients(Message.RecipientType.TO, InternetAddress.parse(recipient[i]));
				Transport.send(message);
			}
\end{lstlisting}

\subsubsection{Ergebnis}
Es werden Empfänger aus einer Text-Datei gelesen und an jede dieser Adressen wird die E-Mail geschickt.

\subsection{Aufgabenteil c)}
\subsubsection{Frage- bzw. Aufgabenstellung}
 Fügen Sie ihrer Mail einen Anhang hinzu (zum Beispiel ein Bild oder PDF-Dokument). Erläutern Sie, in welcher Form der Anhang übertragen wird.

\subsubsection{Lösung}
Zum hinzufügen eines Anhangs wird der Inhalt der E-Mail in 2 Bodys aufgeteilt. Ein Body ist für den Text einer E-Mail. \cite{[7]}
\begin{lstlisting}
			BodyPart msgBodyPart = new MimeBodyPart();
			msgBodyPart.setText(mailText);
			Multipart multipart = new MimeMultipart();
			multipart.addBodyPart(msgBodyPart);
\end{lstlisting}


Der andere Body wird für den Anhang benutzt.
\begin{lstlisting}
			DataSource source = new FileDataSource("C:/Users/Jan Augstein/eclipse-workspace/VSPraktikum3Mail/src/p03.pdf");
			msgBodyPart = new MimeBodyPart();
			msgBodyPart.setDataHandler(new DataHandler(source));
			msgBodyPart.setFileName("P03.pdf");
			multipart.addBodyPart(msgBodyPart);
\end{lstlisting}

Mit \textit{multipart.addBodyPart} werden einzelne BodyParts einem sogenannten MultiPart hinzugefügt. Multipart ist ein Objekt, dass einfach aus mehreren BodyParts besteht und somit den Inhalt der E-Mail bildet. \\
\\
Zum Übertragen vom Anhang wird der \textit{DataHandler} benutzt. DataHandler formt eine Datei (txt, pdf, png) zu einem Byte-Stream um, welcher dann als Anhang übertragen und wieder umgeformt werden kann. \cite{8} \\

\subsubsection{Ergebnis}
Der E-Mail kann ein Anhang hinzugefügt werden.

\section{Quellen}
\begin{thebibliography}{999}

\bibitem {1} Elektronik Kompendium, SMTP - Simple Mail Transfer Protocol \\ \url{https://www.elektronik-kompendium.de/sites/net/0903081.htm}
\bibitem {2} Elektronik Kompendium, POP - Post Office Protocol \\ \url{https://www.elektronik-kompendium.de/sites/net/0903091.htm}
\bibitem {3} Elektronik Kompendium, IMAP - Internet Mail Access Protocol \\ \url{https://www.elektronik-kompendium.de/sites/net/0903101.htm}
\bibitem {4} Mett Heimer, Sending an Email using the JavaMail API \\  \url{http://www.oracle.com/webfolder/technetwork/tutorials/obe/java/javamail/javamail.html}
\bibitem {5} MyKong, JavaMail API – Sending email via Gmail SMTP example, 28. März 2010 \\ \url{https://www.mkyong.com/java/javamail-api-sending-email-via-gmail-smtp-example/}
\bibitem {6} Tutorialspoint, JavaMail API - Authentication \\ \url{https://www.tutorialspoint.com/javamail_api/javamail_api_authentication.htm}
\bibitem {7} SrcCode, Java – Mail mit Attachment versenden, 15.Juni 2011 \\ \url{https://srccode.wordpress.com/2011/06/15/java-mail-mit-attachment-versenden/}
\bibitem {8} Oracle, Class DataHandler \\ \url{https://docs.oracle.com/javaee/5/api/javax/activation/DataHandler.html}

\end{thebibliography}




 
