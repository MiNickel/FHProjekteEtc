\newcommand{\env}[1]{\texttt{#1}}
\newcommand{\command}[1]{\texttt{#1}}
\newcommand{\package}[1]{\texttt{\itshape#1}}
\newcommand{\engl}[1]{(engl: \textit{#1})\xspace}
\newpage

\section*{\LARGE Hinweis zur Erstellung von Praktikumsprotokollen} 
Die Erstellung von Praktikumsprotokollen dient der Vorbereitung sowohl auf die akademische als auch auf die praktische Arbeit. Daher sind die gängigen Regeln des wissenschaftlichen Arbeits einzuhalten. Im Protokoll soll jede Phase des Versuchs bzw. des Praktikums festgehalten und entsprechend dokumentiert werden. Im Folgenden werden die wichtigen Teilaspekte dargestellt und besprochen.

\subsection*{Deckblatt}

Verwenden Sie am Besten, dass bereits vorbereitete Deckblatt. Sollten Sie ein eigenens Deckblatt bevorzugen, sind mindestens die folgenden Einträge zu übernehmen:

\begin{itemize}
	\item Praktikumstitel und Praktikumsnummer
	\item Datum
	\item Versuchsteilnehmer mit Matrikelnummer
	\item Protokollführer (Erste Stelle der Versuchsteilnehmer)
\end{itemize}

\subsection*{Frage- bzw. Aufgabenstellung}
Für jede Teilaufgabe ist die bearbeitete Frage- bzw. Aufgabenstellung unbedingt mit zu übertragen. (Nicht nur die Nummer !)

\subsubsection{Lösungsweg}
Beschreiben Sie die Durchführung der Aufgabenstellung. Hieraus folgt, dass Sie nicht nur die Ergebnisse festhalten, sondern auch den Weg dahin und zwar so kleinschrittig wie nötig und \underline{sinnvoll}. Aufgetretene Effekte bzw. Probleme sind ebenso festzuhalten. Ziel ist es den Versuch für Dritte nachvollziehbar zu gestalten.

\subsubsection{Ergebnisse}
Beschreiben Sie in jedem Fall die erzielten Ergebnisse. Selbst wenn ein Versuch nicht das gewünschte Ergebnis erziehlt hat, sollten Sie diese trotzdem dokumentieren. (\textbf{Für nicht dokumentierte Ergebnisse können keine Punkte vergeben werden})

\subsubsection{Diskussion}
\begin{itemize}
	\item In jedem Fall sollte innerhalb der Diskussion auf die Fragestellung eingegangen werden, jedoch sind auch weitere Aspekte des Versuchs zu diskutieren.
	\newpage
	\item Probleme und besondere Vorkomnisse: Probleme können auftreteten, sollten jedoch in jedem Fall im Protokoll festgehalten werden. Neben der Meldung bei dem entsprechenden Betreuer sind folgende Punkte zu dokumentieren:
	\begin{itemize}
		\item was aufgetreten ist und unter welchen Bedingungen
		\item was Sie versucht haben und mit welchen Ergebnissen
		\item die möglichen Ursachen des Problems
	\end{itemize}
\end{itemize}

\subsubsection{Rückmeldung}
Wir würden uns freuen, wenn Sie das Protokoll auch durchaus nutzen, um Rückmeldung zu den Praktikas zu geben. Gab es Probleme bei der Hardware, ist die Organisation verbesserungswürdig oder gab es unklare Fragestellungen. \\

\textbf{Hinweis}: Sollten wir den Eindruck erlangen, dass Sie das Protokoll nicht eigenständig (in Ihrer Gruppe) angefertigt haben oder dass das Protokoll auf Basis von den Ergebnissen anderer Gruppen angefertigt wurde, wird das entsprechende Praktikum mit 5.0 bewertet und die Abgabe als Täuschungsversuch gesehen.
\\

\textbf{Hinweis:} Neben der Abgabe des Protokolls ist die Teilnahme an Praktikas weiterhin Pflicht und eine notwendige Teilleistung für das Bestehen des Praktikums. In diesen werden Sie Ihre Ergebnisse vorstellen und wir werden diese besprechen.

\newpage
\section{Checkliste für das Praktikumsprotokoll}

\begin{itemize}
	\item[\Square] \textbf{Struktur und Aufbau des Protokolls eingehalten}
	\begin{itemize}
		\item Inhaltsverzeichnis ?
		\item Tabelle mit gelösten und nicht gelösten Aufgaben ?
		\item Für jede Aufgabe: Vorbereitung, Lösungsweg, Ergebnis ?
	\end{itemize}
	\item[\Square] \textbf{Das Protokoll enthält die gesamte, klar formulierte Lösung}
	\begin{itemize}
		\item Vorbedingungen ?
		\item Sourcecode ?
		\item Screenshots ?
		\item Formeln ?
		\item Rechenweg?
		\item Evaluation ?
		\item Erklärungen ?
	\end{itemize}
	\item[\Square] \textbf{Sourcecode als Codesnippets mit erklärendem Text}
	\begin{itemize}
		\item Keine eigene Seite für Sourcecode (Ausnahme: Anhang) ?
		\item Wichtige Codezeilen beschrieben ?
		\item Listings benutzt ?
	\end{itemize}
	\item[\Square] \textbf{Nachweis der Quellen}
	\begin{itemize}
		\item Literaturquellen angegeben und richtig zitiert ?
		\item Wenn Beispielsourcecode verwendet (bspw. aus dem Internet), ist dieser klar erkennbar und referenziert ?
	\end{itemize}
\end{itemize}